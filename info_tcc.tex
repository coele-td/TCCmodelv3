%+++++++++++++++++++++++++++++++++++++++++++++++++++++++++++++++++++++
%Informações do projeto de TCC:
\titulo{Título do trabalho}
\newcommand{\tituloingles}{Título do trabalho em inglês}
\autor{Mévio Albuquerque \\ Caio Isoudo}
\orientador{Prof. Tício de Oliveira}
\coorientador{} %Deixe em branco caso não haja coorientador
\data{\the\year{}} % Apenas o ano
\local{Toledo} % Apenas a cidade

% Insira aqui até cinco palavras chave separadas por ponto. Faça o mesmo para as palavras em inglês. As palavras chave serão automaticamente inseridas no resumo e abstract. O resumo e abstract devem ser aditados no arquivo resumo.tex
\newcommand{\palavraschave}{Palavra 1. Palavra 2. Palavra 3.}
\newcommand{\keywords}{Palavra 1. Palavra 2. Palavra 3.}

% Informações da Banca:

% Entre com os dados dos Membros da banca

% Instituição do(a) Orientador(a)
\newcommand{\oriInst}{UTFPR-TD}

% Primeiro membro:
\newcommand{\membroA}{Prof. Josef Climber} % Nome completo
\newcommand{\membroAinst}{UTFPR-TD} % Instituição

% Segundo Membro:
\newcommand{\membroB}{Prof. Jhonny Epaminomdas} % Nome completo
\newcommand{\membroBinst}{Unicamp} % Instituição

% Data de defesa:
\newcommand{\Data}{ 30 de fevereiro de 2021} 

%+++++++++++++++++++++++++++++++++++++++++++++++++++++++++++++++++++++
%Preambulo (Texto apresentado na contracapa):

% Não alterar o preambulo!:
% Se TCC 1:
%\preambulo{Projeto de Trabalho de Conclusão de Curso apresentado à disciplina de Trabalho de Conclusão de Curso 1 do Curso de Engenharia Eletrônica da Universidade Tecnológica Federal do Paraná - UTFPR Campus Toledo, como requisito parcial para a obtenção do título de Bacharel em Engenharia Eletrônica.}

% Se TCC 2:
%\preambulo{Trabalho de Conclusão de Curso apresentado à disciplina de Trabalho de Conclusão de Curso 2 do Curso de Engenharia Eletrônica da Universidade Tecnológica Federal do Paraná - UTFPR Campus Toledo, como requisito parcial para a obtenção do título de Bacharel em Engenharia Eletrônica.}